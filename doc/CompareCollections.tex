\subsection{Comparing Collections - CompareCollections.py}

The purpose of {\ttfamily CompareCollections.py} is to detect differences 
between two image collections that have been generated from the same source (book, newspaper, etc..).
For every image in the first collection, the tool tries to find a corresponding image in the second collection
in a similar manner as {\ttfamily FindDuplicates.py} does for images in a single collection.



\begin{verbatim}
usage: 
CompareCollections.py [-h] [--threads THREADS] [--filter FILTER]
                      [--sdk SDK] [--nn NN] [--precluster PRECLUSTER]
                      [--clahe CLAHE] [--config CONFIG]
                      [--featdir FEATDIR] [--csv] [--bowsize BOWSIZE]
                      [-v]
                      dir1 dir2
                      {all,extract,duplicates,references,clean}
\end{verbatim}

\paragraph{\ttfamily -{}-threads THREADS}

sets the number of threads to use to extract the features and to create the Bag
of Words. A number of 4 to 8 threads is reasonable, depending on the computer in use.

\paragraph{\ttfamily -{}-filter FILTER}

The filter argument that is passed to {\ttfamily train}. This determines which files in
FEATDIR are treated as feature files. The default is ``.SIFTComparison.feat.xml.gz''.

\paragraph{\ttfamily -{}-sdk SDK}

The number of spatially distinctive keypoints {\ttfamily extractfeatures} will extract from
an image. By default a value of 2000 is used here.

\paragraph{\ttfamily -{}-nn NN}

This option sets the number of nearest neighbours that will be compared to the query
image. 

\paragraph{\ttfamily -{}-precluster PRECLUSTER}

The precluster argument that is passed to {\ttfamily train}.
Number of descriptors to select in precluster-preprocessing (0 = no preclustering).

\paragraph{\ttfamily -{}-clahe CLAHE}

This option enables the preprocessing of the images in a collection with CLAHE 
(contrast limited adaptive histogram equalization) to improve the sharpness
of the images. To be effective, a value greater than 1 (??) has to be supplied.

%\paragraph{\ttfamily -{}-config CONFIG}

\paragraph{\ttfamily -{}-featdir FEATDIR}

With this option set, the tool looks for old feature files in FEATDIR and saves
newly extracted feature files to this directory.

\paragraph{\ttfamily -{}-csv}

Not implemented yet.

\paragraph{\ttfamily -{}-bowsize BOWSIZE}

This option sets the number of ``visual words'' that will be
calculated for the Bag of Words.

The default value is 1000.

\paragraph{\ttfamily -v}

This enables more verbose output.

\paragraph{\ttfamily dir1 dir2}

The two directories that contain the collections to be compared.
